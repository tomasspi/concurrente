\documentclass[12pt,a4paper]{article}
\usepackage[utf8]{inputenc}
\usepackage[spanish]{babel}
\usepackage[T1]{fontenc}
\usepackage{tocbibind} % Bibliografía en el indice
\usepackage{titlesec} % Posibilidad de editar los formatos de chapter
\usepackage{amsmath,amssymb,mathrsfs} % Matemáticas varias
\usepackage{tabularx} % Para las tablas
\usepackage[title]{appendix} % Para anexos
\usepackage[spanish]{babel} % Para modificar labels por defecto
\renewcommand{\baselinestretch}{1} % Interlineado. 1 es estandar
% --- Arreglos varios para la inclusion de imagenes
\usepackage{float}
\usepackage[pdftex]{graphicx}
\usepackage{subfigure}
\usepackage{graphicx}
\graphicspath{{T:/Tom/Facultad/Logos/}}
\usepackage[usenames,dvipsnames]{color}
\DeclareGraphicsExtensions{.png,.jpg,.pdf,.mps,.gif,.bmp}
% --- Para las dimensiones de los márgenes etc
\frenchspacing \addtolength{\hoffset}{-1.5cm}
\addtolength{\textwidth}{3cm} \addtolength{\voffset}{-2.5cm}
\addtolength{\textheight}{4cm}
% --- Para el encabezado
\setlength{\headheight}{33pt}
\usepackage{fancyhdr}
\fancyhead[R]{\includegraphics[height=1cm]{logo_fcefyn_nuevo.jpg}}\fancyhead[L]{\includegraphics[height=1cm]{unc1_a.jpg}}\fancyhead[C]{} \fancyfoot[C]{Página \thepage} \renewcommand{\footrulewidth}{0.4pt}
\pagestyle{fancy}
% --- Para las tablas
\usepackage{multirow} % Juntar filas
\newcolumntype{L}[1]{>{\raggedright\arraybackslash}p{#1}} % Justificar Izq
\newcolumntype{C}[1]{>{\centering\arraybackslash}p{#1}} % Justificar Centrar
\newcolumntype{R}[1]{>{\raggedleft\arraybackslash}p{#1}} % Justificar Der
\usepackage[numbered]{bookmark} % Para que figure las secciones en el PDF
\usepackage{listings} % Para poner código 
\usepackage{enumitem} % Para editar las propiedades de los items
\usepackage{color}
\usepackage[bottom]{footmisc} % Para las notas al pie de la página
\lstset{frame=single} % Código en un cuadro
% --- Para Anexo
\addto\captionsspanish{%
  \renewcommand\appendixname{ANEXO}
  \renewcommand\appendixpagename{ANEXOS}
}
% -------------------------------------------------------- %
% Definicion de colores para el codigo
\lstdefinelanguage{XML}
{
  basicstyle=\ttfamily\footnotesize,
  morestring=[b]",
  moredelim=[s][\bfseries\color{Maroon}]{<}{\ },
  moredelim=[s][\bfseries\color{Maroon}]{</}{>},
  moredelim=[l][\bfseries\color{Maroon}]{/>},
  moredelim=[l][\bfseries\color{Maroon}]{>},
  morecomment=[s]{<?}{?>},
  morecomment=[s]{<!--}{-->},
  commentstyle=\color{DarkOliveGreen},
  stringstyle=\color{blue},
  identifierstyle=\color{red}
}

\renewcommand{\lstlistingname}{Código}

% -------------------------------------------------------- %

\begin{document}

\begin{titlepage}
    \begin{center}
        \vspace*{1cm}
        
        \Large
        \textbf{Universidad Nacional de Córdoba\\
        		Facultad de Ciencias Exactas, Físicas y Naturales}
        
        \vspace{0.5cm}
        \includegraphics[width=0.5\textwidth]{logo_caratula.png}
        
        \vspace{1.5cm}
        
        \textbf{Programación Concurrente\\
        Trabajo Práctico Integrador}
        
        \vfill  
        
        \vspace{0.8cm}
        

        
        \Large
        Navarro, Sebastián\\
        
        \begin{large}
        \href{mailto:navarrosebastian95@gmail.com}{navarrosebastian95@gmail.com}\\
		\end{large} 
		
        Piñero, Tomás Santiago\\
        
        \begin{large}
        \href{mailto:tom-300@hotmail.com}{tom-300@hotmail.com}\\
		\end{large} 
		
        Ingeniería en Computación\\
        Año 2019\\
        
        
    \end{center}
\end{titlepage}
% -------------------------------------------------------- %
% --- Tabla de contenidos

\setcounter{secnumdepth}{1}
\setcounter{tocdepth}{5}
\tableofcontents

% -------------------------------------------------------- %

\newpage
\renewcommand{\baselinestretch}{1}
\setlength{\parskip}{0.5em}

\section{Enunciado}
\label{enunciado}

En este práctico se debe resolver el control de acceso a una playa de estacionamiento con 3 entradas (calles) diferentes. En esta playa hay 2 pisos, y en cada piso pueden estacionar 30 autos. La playa cuenta con 2 salidas diferentes y una única estación de pago (caja). En los accesos a la playa y en los egresos existen barreras que deben modelarse.

La playa cuenta con lugares (3) donde los vehículos se detienen cuando quieren entrar (barrera), una vez que ingresaron se les indica un piso y estacionan (puede ser piso 1 o piso 2). Se debe cuidar que no se permita el ingreso (superar barrera) a más vehículos de los espacios disponibles totales.

Los autos que se retiran de la playa deben liberar un espacio del piso en que se encontraban (diferenciar estacionamiento en cada piso). Cuando un vehículo se va a retirar puede optar por salida a calle 1 o salida a calle 2.
Luego debe abonar la estadía. El cobro de la estadía le lleva a un empleado promedio al menos 3 minutos. (Existe una sola caja)

En caso de que la playa esté llena, se debe encender un cartel luminoso externo que indica tal situación.

El sistema controlador debe estar conformado por distintos hilos, los cuales deben ser asignados a cada conjunto de responsabilidades afines en particular. Por ejemplo: Ingreso de vehículos, manejo de barreras, etcétera.

Debe realizar:
\begin{enumerate}
\item La red de Petri que modela el sistema.
\item Agregar las restricciones necesarias para evitar interbloqueos ni accesos cuando no hay lugar, mostrarlo con la herramienta elegida y justificarlo.
\item Simular la solución en un proyecto desarrollado con la herramienta adecuada (explique porque eligió la herramienta usada).
\item Colocar tiempo en las estación de pago caja (en la/s transición/es correspondiente/s).
\item Hacer la tabla de eventos.
\item Hacer la tabla de estados o actividades.
\item Determinar la cantidad de hilos necesarios (justificarlo)
\item Implementar dos casos de Políticas para:

\begin{itemize}
\item Prioridad llenar de vehículos planta baja (piso 1) y luego habilitar el piso superior. Prioridad salida indistinta (caja).
\item Prioridad llenado indistinta. Prioridad salida a calle 2.
\end{itemize}

\item Hacer el diagrama de clases.
\item Hacer los diagramas de secuencias.
\item Hacer el código.
\item Hacer el testing.
\end{enumerate}

\section{Desarrollo}
\label{desarrollo}

Iniciamos la rdp papi 

\end{document}